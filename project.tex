
\documentclass[conference]{IEEEtran}

\usepackage{amsmath,amssymb,mathabx} 
\usepackage{url}
\usepackage{float}
\usepackage{graphicx}
\usepackage{hyperref} 
\usepackage{listings}
\usepackage{varwidth}
\usepackage{algorithm}
\usepackage{algpseudocode}
\usepackage{array} 
\graphicspath{ {./img/} }

%   ALGORITHM COMMANDS    %
\newcommand{\LineComment}[1]{\Statex /* \hfill \textit{#1} \hfill*/}    % C-style Block Comment
\algrenewcommand{\algorithmiccomment}[1]{\hfill\textit{// #1}}  % C-style comment
\algnewcommand\algorithmicforeach{\textbf{for each}}
\algdef{S}[FOR]{ForEach}[1]{\algorithmicforeach\ #1\ \algorithmicdo}
\renewcommand{\algorithmicrequire}{\textbf{Input:}}
\renewcommand{\algorithmicensure}{\textbf{Output:}}
\hyphenation{op-tical net-works}

\newcommand{\heuristic}{${h(n) = -\frac{\text{\# of targets
surveilled}}{\text{cost}}}$}

\lstset{breaklines}         % auto line break for listings
\hyphenation{op-tical net-works}

\newcommand{\bigoh}[1]{\ensuremath{\mathcal{O}(#1)}}		% define BigOh function

\begin{document} \title{CSCE 686 Complex Optimization\\Algorithm Design Project}

% IEEE Specific formatting
\author{\IEEEauthorblockN{1Lt Chip Van Patten\IEEEauthorrefmark{1}}
        \IEEEauthorblockA{Air Force Institute of Technology\\
                          Dayton, Ohio 45433\\
Email: \IEEEauthorrefmark{1}donald.vanpatten@afit.edu}}
\maketitle

\begin{abstract}
This project addresses a problem in the field of Unmanned Aircraft Systems
(UASs) dealing with assigning UAS orbits.  Given a limited number of UAVs, each
based in specific locations -- each with specific ranges of flight and
operating costs -- and a list of targets to surveil, how do we assign
surveillance orbits to the UASs such that all targets are surveilled while
keeping the UASs close enough to their base of operations that they may safely
return, either when their mission finishes or they've exhausted their fuel, at
minimum cost? The project will attempt to integrate this application domain
with the set cover problem domain and a number of algorithm domains to solve
the problem efficiently.
\end{abstract}

\section{Introduction} \label{intro}

% --Background, layout of report, ...\\
% --PD/AD CSCE686 Software Design Plan with software engineering practice\\
% --Selected problem domain and specific application domain -- high level
% overview\\
% --``Using again software design techniques''

This project addresses a problem in the field of Unmanned Aircraft Systems
(UASs) dealing with assigning UAS orbits to cover a list of targets.  Given a
limited number of UAVs, each assigned to specific units based in specific
locations -- each with specific ranges of flight and operating costs -- and a
list of targets to surveil, how do we assign surveillance orbits to the UASs
such that all targets are surveilled while keeping the UASs close enough to
their base of operations that they may safely return, either when their mission
finishes or they've exhausted their fuel, at minimum cost? The project will
attempt to integrate this application domain with the set cover problem domain
and a number of algorithm domains to solve the problem efficiently.

The rest of this paper is organized as follows.
Section~\ref{sec:designProcess}, discusses the design process starting with the
problem domain (PD) and algorithm domain (AD) and ending with a coded
implementation. In Section~\ref{sec:application}, the problem domain of the UAS
problem discussed briefly in this section is revisted and discussed in greater
detail.  Section~\ref{sec:experiment} describes and analyzes the experiments
run.  Section~\ref{sec:conclusion} concludes with a summary of the project and
brief discussion of related works.

%%    SECTION 2    %%
\section{CSCE 686 Project Algorithmic Design Process} \label{sec:designProcess}

% ``may want to put some details in appendix for each design sub-process''

This section details the algorithmic design process using the top down
disciplined design approach specified in~\cite{lamontDesign, lamontSCP}. The
problem domain is discussed in Section~\ref{sec:problemDomain}. Integration of
the problem domain with the application domain is detailed in
Section~\ref{sec:application}. The algorithm domain is discussed
Section~\ref{sec:initIntegration} where the initial integration of the
algorithm and problem domains begins.  Section~\ref{sec:intermedDesign} refines
the data structures from~\ref{sec:initIntegration} and integrates them into the
program design.  Pseudocode for the algorithm is presented in
Section~\ref{sec:pseudocode} and, finally, an implementation in code based on
the aforementioned pseudocode is described in Section~\ref{sec:code}.




%%    SECTION 2.1    %%
\subsection{\textsc{Problem Domain}} \label{sec:problemDomain}

% PD English discussion
\subsubsection{Problem Domain Discussion}

The set cover problem is a class of problems in which one seeks to select a
minimum number of sets, at minimum cost, so that the union of all selected sets
contains all elements in the input~\cite{stern2006}.  More formally, given a
universe, or ground set of elements, ${\mathcal{R} = \lbrace r_1,\ldots,r_m
\rbrace}$ and a family of sets ${\mathcal{S} = \lbrace S_1,\ldots,S_N
\rbrace}$, a set covering of $\mathcal{R}$ exists if \[ \bigcup\limits_{i =
1}^k S_{j_i} = \mathcal{R},\]

\noindent with $S_{j_i}$ being the covering sets~\cite[pp.
39-40]{christofides1975}. Christofides further state that if, for each $S_j \in
\mathcal{S}$, ``there is associated a (positive) cost $c_j$''~\cite[p.
39]{christofides1975}, then the goal of the set cover problem is to find a set
of sets $\mathcal{S}^\prime$ which includes all $m$ elements of $\mathcal{R}$
while minimizing $\sum_{i = 1}^k c_j$~\cite{christofides1975, stern2006,
talbi2009, wiki:SCP}. 

In the classic, or un-weighted, set cover problem, all ${c_j =
1}$~\cite{stern2006}. In this paper we will only deal with a weighted variant
of the set cover problem, consequently, for the purposes of this paper, the
term \textit{set cover problem} is synonymous with \textit{weighted set cover
problem}, but the former will be used exclusively.

% PD symbolic set development and discussion
\subsubsection{Symbolic Set Development and Discussion} \label{sec:symbol}
We will now develop a symbolic set (adapted from~\cite{christofides1975,
lamontSCP}) which will formalize the notation used throughout this paper and,
hopefully, aid in understanding and clarity as we progress in our development
to the algorithm and application domains.

Domains:
\begin{enumerate}
  \item[] $D_i$ : $(\mathcal{R,S})$
  \item[] $\mathcal{R}$ : set of $m$ elements $r$
  \item[] $\mathcal{S}$ : set of $N$ sets with costs $c$ that may cover
    $\mathcal{R}$
    \begin{enumerate}
      \item[] set cover : $\bigcup\limits_{i=1}^k S_{j_i} =
        \mathcal{R}$~\cite[Eq.~3.12]{christofides1975}\\ 
        -- union of sets cover $\mathcal{R}$
        elements
    \end{enumerate}
  \item[] $D_o$ : set of set-covers $\mathcal{S}^\prime$ of $\mathcal{R}$ with
    additive cost
\end{enumerate}

% PD $0/1$ problem description mapping from set description (utility?)
\subsubsection{$0/1$ Problem Description}
An integer linear program is a special case of integer programming ``in which
unknowns are binary, and only the restrictions must be
satisfied''~\cite{wiki:IP}. According to~\cite{christofides1975}, the set cover
problem ``can be formulated as a [$0/1$] linear program'' as follows:
\begin{align*}
  \text{minimize} \sum\limits_{j=1}^N &c_j \xi_j\\
  \text{subject to} \sum\limits_{j=1}^N &t_{ij} \xi_j \geq 1 &\forall
  r_i \in \mathcal{R}
\end{align*}
where ${c_j \geq 0}$, ${\xi_j \in \lbrace 0,1 \rbrace}$, if ${S_j \notin
\mathcal{S}^\prime}$ or ${S_j \in \mathcal{S}^\prime}$ respectively, and ${t_{ij}
\in \lbrace 0,1 \rbrace}$, if ${r_i \notin S_j}$ or ${r_i \in S_j}$
respectively~\cite[p. 39-40]{christofides1975}.

This is different from the set cover problem defined above in that we encode,
using binary representation, whether every set is in the set cover or not.
We assign a $1$ if the set is in the set cover, and a $0$ otherwise. This
provides a $\log{n}$-approximation algorithm for the minimum
set cover problem~\cite{wiki:SCP}.

% PD Problem Domain complexity (search space, solution space)
\subsubsection{Problem Domain Complexity} \label{sec:pdComplex}

In 1972, Karp, et al. proved that the set cover problem is
NP-Complete~\cite{karp1972}, while the optimization version is
NP-Hard~\cite{wiki:SCP}. According to \cite{wiki:SCP, lamontSCP}, the
complexity of the set cover problem is the powerset of $\mathcal{S}$, that is
${\bigoh{2^{|\mathcal{S}|}} \rightarrow \bigoh{2^N}}$, for both the search and
solution spaces. This is due to the fact that a solution is found, if it
exists, only by searching the space of all possible combinations of
$\mathcal{S}$.

% Problem Domain specification (Input and Output domains, Conditions,...)
\subsubsection{Problem Domain Specification}
Using the symbolic notation developed in Section~\ref{sec:symbol} we define the
input and output domains and conditions for the set cover problem domain.

\begin{enumerate}
  \item[] $D_i$ : $(\mathcal{R,S})$
  \item[] $\mathcal{R}$ : set of $m$ elements $r$
  \item[] $\mathcal{S}$ : set of $N$ sets with costs $c$ that may cover
    $\mathcal{R}$
    \begin{enumerate}
      \item[] set cover : $\bigcup\limits_{i=1}^k S_{j_i} =
        \mathcal{R}$~\cite[Eq.~3.12]{christofides1975}\\ 
        -- union of sets cover $\mathcal{R}$
        elements
    \end{enumerate}
  \item[] $D_o$ : set of set-covers $\mathcal{S}^\prime$ of $\mathcal{R}$ with
    additive cost
\end{enumerate}

Solution:
\begin{enumerate}
  \item[] Minimize additive cover costs subject to following constraints:
    \begin{enumerate}
      \item[-] set cover:\\
        $\sum\limits_{j=1}^N t_{ij} \xi_j \geq 1, \quad i =
        1,2,\ldots,M$~\cite[Eq.  3.14]{christofides1975}\\
        cover all elements at least once
    \end{enumerate}
\end{enumerate}
\medskip



%%    APPLICATION DOMAIN    %%
% --Application Problem  Domain complexity\\
% --Justify selection of approaches as generating a solution(s)\\
% --Mapping of Application to problem domain data structures

\subsection{\textsc{Application Problem Domain}} \label{sec:application}

% Application Problem Domain English discussion
\subsubsection{Application Problem Domain Discussion} \label{sec:APD}

The UAS problem deals with assigning UAS orbits to cover a list of targets.
Given a limited number of UAVs, each assigned to specific units based in
specific locations -- each with specific ranges of distance able to travel and
operating costs -- and a list of targets to surveil, how do we assign
surveillance orbits to the UASs such that all targets are surveilled while
keeping the UASs close enough to their base of operations that they may safely
return, either when their mission finishes or they've exhausted their fuel, at
minimum cost?  The UAS problem relates to the set cover problem in that we are
trying to cover a set of targets with a set of UAVs based on their locations,
all while keeping them within a specified distance of their home stations and
minimizing the associated operating costs. 

Formally, there exists a universe of targets that we want to surveil
${\mathcal{T} = \lbrace t_1,\ldots,t_m \rbrace}$, each with a specific hideout
location $t_{i_h}$, a set of UASs ${\mathcal{U} = \lbrace u_1,\ldots,u_n
\rbrace}$, each with an assigned squadron and operating cost, a set of
squadrons ${\mathcal{F} = \lbrace f_1,\ldots,f_q \rbrace}$ with specific
geographic locations $f_{j_\ell}$, and, finally, a set of schedules
${\mathcal{A} = \lbrace a_1,\ldots,a_r \rbrace}$ for each target which denote
time windows in which the target is able to be surveilled. The objective is to
surveil each target $t$ during their time window $a$ using a minimum number of
UASs ${u \in \mathcal{U}}$ at minimum cost $c$ such that each target is
surveilled.

A review of the available literature reveals that, while much research has been
performed on routing UASs and target coverage, none of the available sources
provide insight into the constrained problem defined in this section. The most
similar research can be found in~\cite{karakaya2014}, which uses a Max-Min Ant
System algorithm, a member of the ant colony optimization algorithm
family~\cite{wiki:aco}. Other research with related, but not similar, goals to
this project are discussed in~\cite{avellar2015, caillouet2010, carlyle2007,
quintero2010, semsch2009, sundar2013}.

% PD complexity
\subsubsection{Application Problem Domain Complexity}

Similar to the problem domain complexity described in
Section~\ref{sec:pdComplex}, the complexity of the application domain is the
powerset of each search space $\bigoh{2^{|\mathcal{T}|} \times
2^{|\mathcal{U}|} \times 2^{|\mathcal{F}|} \times 2^{|\mathcal{A}|}}
\rightarrow \bigoh{2^{m + n + q + r}}$, or every possible combination of
targets, UASs, squadrons, and surveillance time windows.

We now have to prove that the application domain, the UAS problem, is
NP-Complete. This can be shown using the technique described
in~\cite{586textbook} in which the constraints described in
Section~\ref{sec:APD} are relaxed to facilitate a polynomial-time
transformation into the form of a known NP-Complete problem. This would allow a
polynomial-time black-box solver for the known NP-Complete problem, if one
existed, to solve the relaxed-constraint UAS problem. The UAS problem with
relaxed constraints could then be said to be at least as hard as NP-Complete.

Relaxing the constraints of the UAS problem results in a problem as follows:
given a universe, or set of targets, $\mathcal{R}$ a set of UASs $\mathcal{U}$,
and a family of sets $\mathcal{S}$, containing UASs available to surveil
specific targets. Assign to each $s \in \mathcal{S}$ an operating cost $c$. The
goal is then to surveil each target in $\mathcal{R}$ at minimum cost. This
transformation can be done in polynomial time.

The solution returned is easily verified by iterating through $\mathcal{R}$ and
checking that there is in fact a UAS $\in \mathcal{U}$ assigned to surveil that
target. The complexity of the verification is ${\bigoh{|\mathcal{R}| \times
|\mathcal{U}|}}$

Therefore, since we have shownn that, in polynomial-time, the UAS problem can
be transformed to the set cover problem (a known NP-Complete problem),
${\mathit{UAS} \leq_p \mathit{Set\ Cover}}$, and we can verify in
polynomial-time the solution, the UAS problem is NP-Complete.


%%    SECTION 2.2    %%
\section{\textsc{Algorithm Domain/Problem Domain Initial Integrations}} \label{sec:initIntegration}


% Selection of deterministic algorithm
\subsection{Selection of Deterministic Algorithm} \label{sec:detAlg}

The deterministic algorithm used for this project is the A$^*$ search
algorithm. The heuristic function will be defined the same way the classic
greedy approach to this problem is as per~\cite{wiki:SCP}, with the exception
that the heuristic is inverted, as done in~\cite{pjcvphw8}, due to the fact
that A$^*$ chooses the next search node which has the $h(n)$ closest to $0$.
The heuristic function developed for this project is defined as: \heuristic.
The function for deciding which node to visit next is: \[ \text{minimize} f(n)
= h(n) c(n,n^\prime).\]

\subsubsection{Problem Domain/Algorithm Domain Developing Specification}


\subsubsection{Pseudocode}
\emph{Name:} SCP/SPP A$^*$ Search

Algorithm~\ref{alg:astar} is a version of the A$^*$ algorithm detailed
in~\cite{pearl1984}, modified to delay termination to find a complete solution
and to include the heuristic detailed in Section~\ref{sec:detAlg}.

\begin{algorithm}[ht!]
    \begin{algorithmic}[1]
      \LineComment{Initialization}
      \State generate tree
      \State $\mathit{OPEN}$ : priority queue to hold unvisited nodes
      \State $\mathit{CLOSED}$ : set to hold visited nodes
      \State $n$ : a set of sets with properties id, cost, and vector of children
      \Statex
      \State Put the start node $s$ on $\mathit{OPEN}$
      \While{true}
        \LineComment{Solution Function}
        \LineComment{Delayed termination}
        \If{all elements covered \&\& $\mathit{OPEN}$ is empty} \\
        \hskip \algorithmicindent \hskip \algorithmicindent \Return
          \begin{varwidth}[t]{\dimexpr\linewidth-7em}solution
            obtained by tracing back the pointers from $n$ to $s$
          \end{varwidth}
        \EndIf
        \LineComment{Selection Function}
        \State \begin{varwidth}[t]{\dimexpr\linewidth-\algorithmicindent}Remove
          from $\mathit{OPEN}$ and place on $\mathit{CLOSED}$ a set for which
          $f(n) = h(n) + c(n,n^\prime)$ is minimum
          \end{varwidth}
        \Statex
        \LineComment{Next-State-Generator Function}
        \State
          \begin{varwidth}[t]{\dimexpr\linewidth-\algorithmicindent}Otherwise
            expand $n$, generating all its successors, and attach to them
            pointers back to $n$.
          \end{varwidth}
        \Statex
        \LineComment{Feasibility Function}
        \ForEach{successor $n^\prime$ of $n$}
          \LineComment{Objective Function}
          \If{$n^\prime$ is not already on $\mathit{OPEN}$ or $\mathit{CLOSED}$}
            \State Estimate $h(n) = -\frac{\text{\# of elements in the
              set}}{\text{cost}}$
            \State Calculate $f(n^\prime) = h(n^\prime)+c(n,n^\prime)$
          \ElsIf{$n^\prime$ is already on $\mathit{OPEN}$ or $\mathit{CLOSED}$}
            \State \begin{varwidth}[t]{\dimexpr\linewidth-5em}Direct its
              pointers along the path yielding the lowest
            $h(n^\prime)$
            \end{varwidth}
          \EndIf
          \If{\begin{varwidth}[t]{\dimexpr\linewidth-3em}$n^\prime$ required
            pointer adjustment \&\& was on $\mathit{CLOSED}$}
          \end{varwidth}
            \Statex
            \State Put $n^\prime$ on $\mathit{OPEN}$
          \EndIf
        \EndFor
      \EndWhile
    \end{algorithmic}
    \caption{A$^*$ Search Algorithm from~\cite{pearl1984} modified with a
    heuristic and delayed termination to solve SCP/SPP}
    \label{alg:astar}
\end{algorithm}


% Selection of stochastic algorithm
\subsection{Selection of Stochastic Algorithm}
For the simulated annealing

``This theorem states that no search algorithm is better than a random search
on the space of all possible problems —in other words, if a particular
algorithm does better than a random search on a particular type of problem, it
will not perform as well on another type of problem, so that all in all, its
global performance on the space of all possible problems is equivalent to a
random search.''~\cite{collet2007, wiki:nfl}

% Discuss flow of design process; overview of Algorithm/Program Design
%   Methodology
\subsubsection{Algorithm/Program Design Methodology}



% To show design process reflected in specific PD/AD refinement with data
%   structures
\subsubsection{Refinement of PD/AD with Data Structures}



%%    SECTION 2.3    %%
\subsection{\textsc{Algorithm Design Specifications}} \label{sec:designSpec}


% Integrating Problem Domain with Algorithm Domain towards program
% Define AD structure using PD data structures ``generally difficult''
\subsubsection{Integration of Problem and Algorithm Domains}



% Instantiation of general Algorithm Domain specification search constructs with
%   Program Domain data structures for development of Program Specification
% Discussion of high-level design process -- towards efficient data structures
%   & program

%%    SECTION 2.4    %%
\subsection{\textsc{Intermediate Algorithm Designs}} \label{sec:intermedDesign}

% Refine data structures (justify each) from initial Program Domain
%   data structures
% Integrate refined data structures into evolving  program design structure for
%   each Approach.  Selection of ADT’s for data structure refinement

%%    SECTION 2.5    %
\subsection{\textsc{Algorithm Pseudocode}} \label{sec:pseudocode}

% Refine intermediate data structures and program design
% Discussion of low-level design process – toward efficient data structures and
%   Program development of program functional/OOD structure for each approach

%%    SECTION 2.6    %%
\subsection{\textsc{Program Implementation}} \label{sec:code}

% --Selection of programming language and programming environment\\
% --Code implementation- mapping discussion\\
% --Functional/Object-Oriented structure chart? (comment with the search
% constructs)\\
% --Order-of analysis (complexity analysis) of code\\
% --Address code development using “good” software engineering principles
% (refs)\\
% --“Add comments/prose”
The implementation for this project (along with the {\LaTeX} files for this
document) can be seen on GitHub\footnote{https://github.com/chipvp/686Project}.

%%    SECTION 4    %%
\section{Design of Experiments} \label{sec:experiment}

% --(class handout for evaluations; M\&F, Appendix B; Barr, Talbi, …)\\
% --Objectives of experiment (selected application and general)\\
% --Tests and Results (application, benchmarks; increasing dimensions – from
% web?)\\
% --Validation of code for each approach and application solution(s)\\
% --Search Graph (partial) for your application (DFS with limited BT or BFS)\\
% ----Algorithm state execution listing (partial)\\
% --Discuss your  graph search diagram related to application\\
% --Analysis of Results (relate to application) – graphs, tables, stats,...\\
% --Briefly compare to other algorithm designs and results (paper review in
% app.))\\
% --Relate experiments to benchmarks and results found on web, books,...\\

Experiments for this project were designed in accordance with the guidelines
set forth in~\cite{barr2001}. An Apple MacBook Pro equipped with a 2.8 GHz
Intel Core i7 processor with 16 GB of 1600 MHz DDR3 RAM running Mac OS X
version 10.11.4 was used to run all experiments.

%%    SECTION 5    %%
\section{Conclusions} \label{sec:conclusion}

% --Relate your  program design experience to educational objectives\\
% --Briefly discuss experimental design results, analysis and comparisons.\\
% --Summary evaluation of  using  of "good” software engineering principles\\
% ----quality of algorithm (effective, efficient)\\
% ----quality of implementation \\
% --Briefly discuss use of other  algorithms and software (web, texts, …)\\
% --As appropriate, relationship to general Boolean matrix problems
% (references)\\ 
% --SCP, SAT, …

% use section* for acknowledgment
% \section*{Acknowledgment}
% The authors would like to thank...

% trigger a \newpage just before the given reference
% number - used to balance the columns on the last page
% adjust value as needed - may need to be readjusted if
% the document is modified later
%\IEEEtriggeratref{8}
% The "triggered" command can be changed if desired:
%\IEEEtriggercmd{\enlargethispage{-5in}}

\bibliography{global.bib}{}
\bibliographystyle{plain}

\end{document}
